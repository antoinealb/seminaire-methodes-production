% Déclaration du type de document (report, book, paper, etc...)
\documentclass[a4paper]{report} 
 
% Package pour avoir Latex en français
\usepackage[utf8]{inputenc}
\usepackage[frenchb]{babel}
 
% Quelques packages utiles
\usepackage{listings} % Pour afficher des listings de programmes
\usepackage{graphicx} % Pour afficher des figures
\usepackage{amsthm}   % Pour créer des théorèmes et des définitions
\usepackage{amsmath}
\usepackage{microtype} % Optical margins FTW
\usepackage{url}
\usepackage{booktabs} % Allows the use of \toprule, \midrule and \bottomrule in tables for horizontal lines
\usepackage{siunitx}
\usepackage{floatrow}
\usepackage{caption}
\usepackage{subcaption}


% Auteur
\author{Antoine Albertelli}
 
% Titre du document
\title{Rapport de stage d'usinage}

% Début du document
\begin{document}

\input{titlepage.tex}
\tableofcontents

\begin{figure}
        \centering
        \begin{subfigure}[b]{0.4\textwidth}
                \includegraphics[width=\textwidth]{two-shots-a}
                \caption{Remplissage simultané des moules}
                \label{fig:gull}
        \end{subfigure}%
        ~ 
        \begin{subfigure}[b]{0.4\textwidth}
                \includegraphics[width=\textwidth]{two-shots-b}
                \caption{Rotation des moules}
                \label{fig:lol}
        \end{subfigure}%
        \caption{Procedé two-shot}\label{fig:animals}

        \floatfoot{Source: \url{http://www.sumitomo-shi-demag.eu/processes/multi-component-technology/rotary-plate.html}}
\end{figure}

\appendix
\listoffigures

\nocite{*} % tells bibtex to include everything
\bibliographystyle{abbrv-fr}
\bibliography{biblio}
\end{document}
