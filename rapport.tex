% Déclaration du type de document (report, book, paper, etc...)
\documentclass[a4paper]{paper} 
 
% Package pour avoir Latex en français
\usepackage[utf8]{inputenc}
\usepackage[frenchb]{babel}
 
% Quelques packages utiles
\usepackage{listings} % Pour afficher des listings de programmes
\usepackage{graphicx} % Pour afficher des figures
\usepackage{amsthm}   % Pour créer des théorèmes et des définitions
\usepackage{amsmath}
\usepackage{microtype} % Optical margins FTW
\usepackage{url}
\usepackage{booktabs} % Allows the use of \toprule, \midrule and \bottomrule in tables for horizontal lines
\usepackage{siunitx}
\usepackage{floatrow}
\usepackage{caption}
\usepackage{subcaption}
\usepackage{mhchem}

% Début du document
\begin{document}

\input{titlepage.tex}
\tableofcontents

\clearpage
\section{Introduction}
Un \gls{mid}  est un circuit électronique
dont le support est un substrat thermoplastique moulé par injection, par
opposition aux \glspl{pcb} conventionnels dont le substrat est un composite
plat.

Les \glspl{mid} permettent de regrouper dans une seule pièce des fonctions
mécaniques et électroniques. En effet, contrairement aux \glspl{pcb}, les
\glspl{mid} peuvent être conçus en trois dimensions, ce qui réduit
considérablement le nombre de composants et de connecteurs, diminuant ainsi le
temps d'assemblage, et donc le coût du système. 

Les \glspl{mid} ne sont toutefois pas une solution de remplacement des
\glspl{pcb}, car ils ne permettent pas une grande densité de pistes, tandis que
les \glspl{pcb} à plus de deux couches sont désormais peu coûteux et
permettent des circuits à forte densité.


\section{Applications}
Les \glspl{mid} se rencontrent dans des domaines très variés, mais qui ont tous en commun de demander une forte intégration entre l'électronique et la mécanique.

\subsection{Automobile}
L'électronique embarquée dans les voitures devient de plus en plus complexe afin d'augmenter le confort et la sécurité des passagers. 
À l'inverse, le coût d'assemblage, le poids et par conséquent le nombre de pièces doivent être réduits afin de continuer à proposer des modèles abordables.
Les \glspl{mid} sont des solutions adaptées pour remplacer les fils et la connectique, tout en intégrant des fonctions mécaniques, comme des boutons ou de la structure.

À la fig. \ref{fig:mid-automotive-example}, on peut remarquer le gain en simplicité obtenu dans le cas d'un volant multi-fonction.
Ces volants, de plus en plus courants dans les voitures actuelles, permettent au conducteur de changer de chaîne de radio, de rapport de boîte à vitesse, etc.
L'utilisation d'un \gls{mid} a permis ici de simplifier le câblage: Les fils ont été supprimés et les boutons remplacés par des contacts recouvert d'un tapis conducteur en silicone.

\begin{figure}[h]
        \centering
        \begin{subfigure}[t]{0.4\textwidth}
                \includegraphics[width=\textwidth]{images/conventional_steering_wheel}
                \caption{Volant multi-fonction conventionnel.}
                \label{fig:conventional-wheel}
        \end{subfigure}%
        ~ 
        \begin{subfigure}[t]{0.4\textwidth}
                \includegraphics[width=\textwidth]{images/mid_steering_wheel}
                \caption{Volant multi-fonction \gls{mid} (vue \textsc{cao})}
                \label{fig:mid-wheel}
        \end{subfigure}
        \caption{Exemple d'utilisation des \glspl{mid} dans l'automobile.}\label{fig:mid-automotive-example}
\end{figure}

\subsection{Médical}
Une autre application intéressante des \glspl{mid} se situe dans le milieu médical, et plus particulièrement des implants auditifs.
Ce secteur cherche en effet à miniaturiser de plus en plus ses produits, afin de les rendre plus discrets, ce qui demande une intégration entre électronique et mécanique très poussée.
La fig. \ref{fig:mid-siemens-example} montre l'utilisation faite par Siemens d'un \gls{mid} comme châssis et connecteur d'un implant.
Il n'existe, à notre connaissance, pas d'exemple d'utilisation de \gls{mid} dans des prothèses vitales, comme des stimulateurs cardiaques.


\begin{figure}[h]
    \begin{center}
        \includegraphics[width=0.3\textwidth]{images/mid-hearing-aid_RED}
        \caption{Chassis d'implant auditif, réalisé en \gls{mid}.}
        \label{fig:mid-siemens-example}
    \end{center}
\end{figure}

\subsection{Consumer Electronic}
Dans le domaine de l'électronique grand public, la miniaturisation, la diversification et la réduction des coûts posent des difficultés importantes.
Par exemple, créer une antenne miniaturisée en \textsc{mid} offre deux avantages majeurs par rapport aux antennes classiques et aux \textit{\glspl{pcb} trace antenna} :
\begin{itemize}
    \item L'encombrement est réduit, soit en enroulant l'antenne sur elle même, soit en l'intégrant directement au chassis de l'appareil,
    \item La directivité de l'antenne est parfaitement maîtrisable, permettant ainsi une réception optimale quelle que soit l'orientation.
\end{itemize}
La fig. \ref{fig:mid-ce-example} montre l'utilisation des \glspl{mid} pour fabriquer une antenne miniature (\SI{4}{\milli\meter} de côté) soudable sur un \gls{pcb} et pour fabriquer une antenne intégrée au chassis d'un smartphone.
Ces images proviennent de la société Molex, qui utilise les \glspl{mid} pour la production à très grande échelle et le prototypage de ses antennes.

\begin{figure}[h]
        \centering
        \begin{subfigure}[t]{0.4\textwidth}
                \includegraphics[width=\textwidth]{images/mid-antenna}
                \caption{Antenne soudable sur \gls{pcb}.}
        \end{subfigure}%
        ~ 
        \begin{subfigure}[t]{0.4\textwidth}
                \includegraphics[width=\textwidth]{images/lds-molex-antenna}
                \caption{Antenne intégrée au boîtier.}
        \end{subfigure}
        \caption{Utilisation des \glspl{mid} dans les antennes.}\label{fig:mid-ce-example}
\end{figure}



\section{Conception de MID}
La conception des \gls{mid} commence par la création du schéma électrique dans un logiciel de conception électronique standard, puis le schéma est exporté sous forme de \emph{netlist}, c'est à dire d'une liste de connexion entre différents composants.
Le design mécanique du \gls{mid} se fait dans un logiciel de \textsc{dao} standard, comme Solidworks puis est exporté au format \textsc{step}.

La \emph{netlist} et le fichier \textsc{step} sont ensuite importés dans un logiciel spécifique \footnote{Nextra de chez Mecadtron \url{http://www.mecadtron.com/produkte/nextra.en.php}} pour l'étape de routage (placement des pistes).
Lors de cette étape on place aussi des repères appelés \emph{fiducials} qui seront utilisés par les machines d'activation et d'assemblage pour faire un alignement visuel.
On peut également placer des codes-barres 2D (\emph{datamatrix}) qui seront modifiés par la machine d'activation pour y placer un numéro de série, afin de faire du suivi des pièces.

\subsection{Matériaux utilisables}
Les contraintes inhérentes à la technologie \gls{lds} limitent fortement le choix des matériaux :
\begin{itemize}
    \item La pièce étant injectée puis activée, le matériaux choisi doit être un thermoplastique,
    \item Pour l'étape d'activation, le polymère doit pouvoir se lier avec un dopant organométallique, généralement à base de palladium,
    \item L'assemblage des composants par \textit{reflow soldering} exige une température de fusion élevée.
\end{itemize}

LPKF a donc créé une liste de matériaux utilisables et reccomandés pour le \gls{lds}.
Cette liste est visible dans le tab. \ref{tab:mid-materials}.
Une liste de fournisseur proposant ces matériaux déjà dopés est également disponible chez LPKF \cite{mid-design-rules}.

% Table des matériaux
\begin{table}[h]
\centering
\begin{tabular}{l l}
\toprule 
Type & Matériau \\
\midrule % In-table horizontal line
LCP & Liquid Crystal Polymer \\
PA 6/6T & Polyamide \\
PBT & Polytéréphtalate de butylène \\
PBT/PET & Mélange de PBT et de PET \\
PPA & Polyphthalamide \\
PC & Polycarbonate \\
PC/ABS & Mélange de PC et d'ABS \\ 
\bottomrule 
\end{tabular}
\caption{Matériaux utilisables comme substrat}
\label{tab:mid-materials}
\floatfoot{Source: \cite{mid-design-rules}}
\end{table}

\subsection{Limites du procédé LDS}
La conception de circuits \gls{mid} est délicate car elle doit prendre en compte des limites provenant de l'injection plastique, de l'activation laser et de la métallisation.
L'ingénieur doit donc porter une attention particulière aux détails lors de la conception, car des petits détails, comme des trous borgnes, peuvent augmenter significativement le coût de la pièce, voir même la rendre impossible à réaliser.
L'ensemble des règles de conception à respecter est compilée dans un seul document, fourni par LPKF \cite{mid-design-rules}.
Nous allons ici nous concentrer sur les plus importantes.

La plus grosse limitation des \gls{mid} par rapport au \gls{pcb} est la faible densité de pistes atteignable.
En effet, il est impossible de fabriquer des \gls{mid} avec des couches internes, là où des circuits à 16 couches sont facilement atteignables dans le domaine des \gls{pcb}.
De plus, les \glspl{mid} ne permettent pas de faire des pistes très fines ou très serrées : On considére généralement qu'il faut rester au dessus de \SI{150}{\micro\meter} .

\textbf{A compléter\ldots}



\section{Procédés de fabrication des MID}
Il existe actuellement 2 méthodes de production de \textsc{mid} sur le marché :
le \emph{Laser Direct Structuring} et le \emph{Two-shot molding}. Le premier est
de loin le plus répandu (environ 80\% du marché selon \cite{mid-2011}), car moins
cher et plus adapté aux petites séries, tandis que le second est utilisé lorsque
la géométrie de la pièce ne permet pas l'usage du \textsc{lds}.

\subsection{Laser Direct Structuring}
\begin{figure}[h]
    \begin{center}
        \includegraphics[width=\textwidth]{images/lds_process}
        \caption{Vue d'ensemble du processus \emph{Laser Direct Structuring}}\label{fig:lds-process}
    \end{center}
\end{figure}


\begin{figure}
        \centering
        \begin{subfigure}[b]{0.4\textwidth}
                \includegraphics[width=\textwidth]{images/two-shots-a}
                \caption{Remplissage simultané des moules}
        \end{subfigure}%
        ~ 
        \begin{subfigure}[b]{0.4\textwidth}
                \includegraphics[width=\textwidth]{images/two-shots-b}
                \caption{Rotation des moules}
        \end{subfigure}%
        \caption{Procedé two-shot}\label{fig:animals}

    \floatfoot{Source: \cite{sumitomo-multicomponent}}
\end{figure}

\appendix

\clearpage
\listoffigures

\nocite{*} % tells bibtex to include everything
\bibliographystyle{abbrv-fr}
\bibliography{biblio}
\end{document}
