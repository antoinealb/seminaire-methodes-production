
\section{Conception de MID}
La conception des \textsc{mid} commence par la création du schéma électrique dans un logiciel de conception électronique standard, puis le schéma est exporté sous forme de \emph{netlist}, c'est à dire d'une liste de connexion entre différents composants.
Le design mécanique du \textsc{mid} se fait dans un logiciel de \textsc{dao} standard, comme Solidworks puis est exporté au format \textsc{step}.

La \emph{netlist} et le fichier \textsc{step} sont ensuite importés dans un logiciel spécifique \footnote{Nextra de chez Mecadtron \url{http://www.mecadtron.com/produkte/nextra.en.php}} pour l'étape de routage (placement des pistes).
Lors de cette étape on place aussi des repères appelés \emph{fiducials} qui seront utilisés par les machines d'activation et d'assemblage pour faire un alignement visuel.
On peut également placer des codes-barres 2D (\emph{datamatrix}) qui seront modifiés par la machine d'activation pour y placer un numéro de série, afin de faire du suivi des pièces.

\subsection{Matériaux utilisables}
Les matériaux utilisés en \textsc{mid} % TODO

% Table des matériaux
\begin{table}[h]
\centering
\begin{tabular}{l l}
\toprule 
Type & Matériau \\
\midrule % In-table horizontal line
LCP & Liquid Crystal Polymer \\
PA 6/6T & Polyamide \\
PBT & Polytéréphtalate de butylène \\
PBT/PET & Mélange de PBT et de PET \\
PPA & Polyphthalamide \\
PC & Polycarbonate \\
PC/ABS & Mélange de PC et d'ABS \\ 
\bottomrule 
\end{tabular}
\caption{Matériaux utilisables comme substrat}
\label{tab:mid-materials}
\floatfoot{Source: \cite{mid-design-rules}}
\end{table}
