
\section{Conception de MID}
La conception des \gls{mid} commence par la création du schéma électrique dans un logiciel de conception électronique standard, puis le schéma est exporté sous forme de \emph{netlist}, c'est à dire d'une liste de connexion entre différents composants.
Le design mécanique du \gls{mid} se fait dans un logiciel de \textsc{dao} standard, comme Solidworks puis est exporté au format \textsc{step}.

La \emph{netlist} et le fichier \textsc{step} sont ensuite importés dans un logiciel spécifique \footnote{Nextra de chez Mecadtron \url{http://www.mecadtron.com/produkte/nextra.en.php}} pour l'étape de routage (placement des pistes).
Lors de cette étape on place aussi des repères appelés \emph{fiducials} qui seront utilisés par les machines d'activation et d'assemblage pour faire un alignement visuel.
On peut également placer des codes-barres 2D (\emph{datamatrix}) qui seront modifiés par la machine d'activation pour y placer un numéro de série, afin de faire du suivi des pièces.

\subsection{Matériaux utilisables}
Les contraintes inhérentes à la technologie \gls{lds} limitent fortement le choix des matériaux :
\begin{itemize}
    \item La pièce étant injectée puis activée, le matériaux choisi doit être un thermoplastique,
    \item Pour l'étape d'activation, le polymère doit pouvoir se lier avec un dopant organométallique, généralement à base de palladium,
    \item L'assemblage des composants par \textit{reflow soldering} exige une température de fusion élevée.
\end{itemize}

LPKF a donc créé une liste de matériaux utilisables et reccomandés pour le \gls{lds}.
Cette liste est visible dans le tab. \ref{tab:mid-materials}.
Une liste de fournisseur proposant ces matériaux déjà dopés est également disponible chez LPKF \cite{mid-design-rules}.

% Table des matériaux
\begin{table}[h]
\centering
\begin{tabular}{l l}
\toprule 
Type & Matériau \\
\midrule % In-table horizontal line
LCP & Liquid Crystal Polymer \\
PA 6/6T & Polyamide \\
PBT & Polytéréphtalate de butylène \\
PBT/PET & Mélange de PBT et de PET \\
PPA & Polyphthalamide \\
PC & Polycarbonate \\
PC/ABS & Mélange de PC et d'ABS \\ 
\bottomrule 
\end{tabular}
\caption{Matériaux utilisables comme substrat}
\label{tab:mid-materials}
\floatfoot{Source: \cite{mid-design-rules}}
\end{table}

\subsection{Limites du procédé LDS}
La conception de circuits \gls{mid} est délicate car elle doit prendre en compte des limites provenant de l'injection plastique, de l'activation laser et de la métallisation.
L'ingénieur doit donc porter un soin tout particulier à la conception, car des petits détails, comme des trous borgnes, peuvent augmenter significativement le coût de la pièce, voir même la rendre impossible à réaliser.
L'ensemble des règles de conception à respecter est compilée dans un seul document, fourni par LPKF \cite{mid-design-rules}.
Nous allons ici nous concentrer sur les plus importantes.

La plus grosse limitation des \gls{mid} par rapport au \gls{pcb} est la faible densité de pistes atteignable.
En effet, il est impossible de fabriquer des \gls{mid} avec des couches internes, là où des circuits à 16 couches sont facilement atteignables dans le domaine des \gls{pcb}.
De plus, les \glspl{mid} ne permettent pas de faire des pistes très fines ou très serrées : On considére généralement qu'il faut rester au dessus de \SI{150}{\micro\meter} .

\textbf{A compléter\ldots}


\subsection{Prototypage}
Le LDS offre différentes possibilités de prototypage simples, puisque la seule étape différente de la production est la 
création de la pièce de base en polymère. Les étapes d'activation et de métallisation sont donc inchangées par rapport à la production.

\begin{itemize}
    \item Une première méthode consiste à usiner directement dans un bloc de polymère activable (les déchets peuvent être réutilisés pour l'injection
        d'autres pièces). Pour cette alternative, il faut posséder une CNC pour effectuer l'usinage dans le bloc.


    \item Une deuxième méthode est de mouler une pièce à l'aide d'un "soft mold". Il s'agit d'un moule, typiquement en alu, produit grâce à l'usinage
        grande vitesse, qui permet de garantir une série d'environ 1000 pièces. Il faut savoir que ce moule doit être produit par une entreprise spécialisée, donc
        le délai avant la réception du moule et le début du prototypage, est d'environ 4 semaines. Après réception du moule, le processus est identique à la méthode
        standarde : injection, activation laser, métalisation. Le coût de cette méthode représente 10\% du coût de la pièce finale.

    \item Il est également possible de créer un moule en silicone à partir de l'empreinte d'une pièce maîtresse stéréolitographiée (ou imprimée en 3D). Ce moule assure 
        une série de 25 pièces coulées sous vide en PUR activable, mais nécessite une machine de coulage sous vide. 

    \item LPKF a également développé et vend un spray de revêtement activable par laser. L'avantage de ce revêtement est qu'il est applicable sur une pièce en polymère
        classique, rendant ainsi la surface de cette pièce activable au laser, après un passage bref au four pour solidifier la laque. 
        Il est donc possible de créer la pièce de base par impression 3D, en choisissant un polymère stable à \SI{90}{\degree} pour supporter les bains de métallisation. La suite du 
        processus est également identique à la production standarde. Le délai de réception d'une pièce imprimée en 3D commandée dans une entreprise spécialisée
        est d'environ 1 semaine.
\end{itemize}









