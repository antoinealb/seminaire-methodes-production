\section{Introduction}
Un \emph{Molded Interconnect Device} (\textsc{mid}) est un circuit électronique
dont le support est un substrat thermoplastique moulé par injection, par
opposition au \textsc{PCB} conventionnels dont le substrat est un composite
plat.

Les \textsc{mid} permettent de regrouper dans une seule pièce des fonctions
mécaniques et éléctroniques. En effet, contrairement au \textsc{pcb} conventionnels, les
\textsc{mid} peuvent être conçus en trois dimensions, ce qui réduit
considérablement le nombre de composants et de connecteurs, diminuant ainsi le
temps d'assemblage, et donc le coût du système. 

Les \textsc{mid}s ne sont toutefois pas une solution de remplacement des
\textsc{pcb}s car ils ne permettent pas une grande densité de pistes, tandis que
les \textsc{pcb}s à plus de deux couches sont désormais peu coûteux et
permettent des circuits à forte densité.
