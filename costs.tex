\section{Étude de coûts}
Pour notre étude de coûts, nous avons décidé de nous baser sur la série en production chez Cicorel lors de notre visite.
Les tableaux de calculs de coûts restent néanmoins valables, moyennant un changement de certaines valeurs.
Les paramètres à changer en fonction de la pièce sont indiqués dans le détail de chaque partie.


\paragraph{Descriptif de la pièce}
La pièce que nous considérons est une pièce relativement simple.
Il s'agit d'un couvercle anti-ingénierie inverse pour lecteur de carte de crédit.
Le rôle de cette pièce est de détecter toute tentative d'accéder à l'intérieur de l'appareil, soit en le démontant, soit en le perçant.
Pour accomplir cette fonction, deux pistes sont placées en serpentin à l'intérieur du couvercle.
L'appareil est programmé pour effacer toute les données sensibles dès qu'une de ces pistes est rompue.
Cette pièce était malheureusement confidentielle et nous n'avons pas pu la prendre en photo.
On peut l'approximer par la boîte creuse visible à la fig. \ref{fig:example-part}.

\begin{figure}[h]
    \begin{center}
%        \missingfigure[figwidth=\textwidth]{Image de la pièce d'exemple.}
        \includegraphics[width=\textwidth]{images/example_part/example_mid}
        \caption{Pièce d'exemple. Les dimensions et la géométrie sont approximatives.}\label{fig:example-part}
    \end{center}
\end{figure}
\subsection{Injection}
\begin{table}[h!]
\centering 
\begin{tabular}{l S[table-format=3.2] r} 
\toprule 
\multicolumn{2}{l}{\textbf{Électricité}} & \\ 
Consommation électrique & 5 & \si{\kilo\watt} \\
Prix de l'électricité & 0.2 & \si{\chf\per\kilo\watt\per\hour} \\
\cmidrule(l){2-3}
Coût d'électricité & 1 & \si{\chf\per\hour} \\
\midrule
\multicolumn{2}{l}{\textbf{Machines}} & \\ 
Machine d'injection & 130000 & \si{\chf} \\
Fonctionnement & 6240 & \si{\hour\per\annee} \\
\cmidrule(l){2-3}
Amortissement sur 5 ans & 4.17 & \si{\chf\per\hour} \\
\midrule
\multicolumn{2}{l}{\textbf{Opérateurs}} & \\ 
Opérateur à 10\% & 6& \si{\chf\per\hour} \\
\midrule

\multicolumn{2}{l}{\textbf{Matière}} & \\ 
Matière (\textsc{pc}) & 11 & \si{\chf\per\kilogram} \\ 
Poids de la pièce & 25 & \si{\gram} \\
\cmidrule(l){2-3}
Coût de la matière & 0.27 & \si{\chf\per\piece} \\

\midrule
\multicolumn{2}{l}{\textbf{Outillage}} & \\ 
Prix du moule & 20000 & \si{\chf} \\
Pièces par série & 500000& \si{\piece} \\
\cmidrule(l){2-3}
Coûts d'outillage & 0.04 & \si{\chf\per\piece} \\

\midrule
\midrule
Coût horaire & 11.17  & \si{\chf\per\hour} \\
Temps de cycle & 20 & \si{\second\per\piece} \\
\cmidrule(l){2-3}
\textbf{\textsc{Total}} & 0.38& \si{\chf\per\piece} \\

\bottomrule 
\end{tabular}
\caption{Calcul des coûts de l'injection plastique} 
\label{tab:cost-molding}
\end{table}



L'étape d'injection plastique demande un moule usiné soit par éléctroérosion, soit par usinage à grande vitesse.
Le coût du moule a été estimé suivant les calculs présentés dans le séminaire\cite{electroerosion-2013}.
Le détail de ces calculs sortant du cadre du présent séminaire, nous avons fait le choix de ne pas les inclure. 

Nous avons également supposé que la pièce est faite dans la matière la moins cher, c'est à dire le polycarbonate.
Les polymères activables sont environ 20\% plus cher que leur homologue non activable d'après notre contact chez Cicorel.
Les prix des matériaux varient entre \SI{11}{\chf\per\kilogram} pour du \textsc{pc} et \SI{220}{\chf\per\kilogram} pour du \textsc{peek}.

\subsection{Activation Laser}
Pour calculer le coût de l'étape d'activation sélective (\gls{lds}) nous avons fait quelques hypothèses :
\begin{itemize}
    \item Le secteur d'activation laser tourne \SI{24}{\hour} par jour, 5 jours sur 7.
    \item La machine est une Microline 160i de chez LPKF coûtant \SI{220000}{\chf}.
        Elle est amortie en 5 ans.
    \item Un opérateur ne peut s'occuper que d'une machine à la fois.
    \item La machine consomme en permanence sa puissance de pointe soit \SI{2.5}{\kilo\watt} \cite{lpkf-microline-series}.
        L'électricité représentant une fraction faible du coût final, cette hypothèse n'induis pas de grande erreurs.
\end{itemize}

Le calcul de la surface occuppée par les pistes sur notre circuit se base sur une largeur de piste de \SI{150}{\micro\meter} et une
distance entre deux pistes de \SI{150}{\micro\meter} également, pour une piste occupant toute la surface du circuit.
Pour une autre pièce, la surface des pistes peut souvent être obtenue dans le logiciel de conception.

\begin{table}[h!]
\centering 
\begin{tabular}{l S[table-format=3.2] r} 
\toprule 
\multicolumn{2}{l}{\textbf{Électricité}} & \\ 
Consommation électrique & 2.5 & \si{\kilo\watt} \\
Prix de l'électricité & 0.2 & \si{\chf\per\kilo\watt\per\hour} \\
\cmidrule(l){2-3}
Prix de l'électricité & 0.5 & \si{\chf\per\hour} \\
\midrule
\multicolumn{2}{l}{\textbf{Machines}} & \\ 
LPKF Microline 160i & 220000 & \si{\chf} \\
Fonctionnement & 6240 & \si{\hour\per\annee} \\
\cmidrule(l){2-3}
Amortissement sur 5 ans& 7.05 & \si{\chf\per\hour} \\
\midrule
\multicolumn{2}{l}{\textbf{Opérateurs}} & \\ 
Opérateur à 100\% & 60& \si{\chf\per\hour} \\
\midrule
\midrule
Coût horaire & 67.68 & \si{\chf\per\hour} \\
Temps de cycle & 25 & \si{\second\per\piece} \\
\cmidrule(l){2-3}
\textbf{\textsc{Total}} & 0.47 & \si{\chf\per\piece} \\

\bottomrule 
\end{tabular}
\caption{Calcul des coûts de l'activation sélective par laser} 
\label{tab:cost-laser-activation}
\end{table}


\subsection{Métallisation}
Calculer le coût réel de cette étape est difficile, car beaucoup d'informations ne sont pas disponibles publiquement : le prix des bains et de la \gls{step} pour ne citer que ces deux là, ne sont pas trouvable chez les fournisseurs.
De plus filtrer les bains pour les recycler permet de récupérer un peu de matière qui est revendue, ce qui intervient dans le calcul de coût, mais est difficile à estimer.
Finalement, notre contact chez Cicorel a refusé de détailler leur méthode de calculs pour ces bains, en nous répondant qu'il s'agissait de données internes et confidentielles.
Nous avons par contre pu obtenir le coût de l'étape de métallisation pour les pièces en cours de production : \SI{0.70}{\chf\per\piece}.

Le changement des bains se faisant à fréquence fixe, et la durée de séjour des pièces dans chaque bain ne dépendant pas du type de pièce, le coût de l'étape de métallisation de dépend donc que du nombre de pièce
pouvant entrer simultanément dans la cuve.
Chaque rack à couvercle pouvait contenir 80 pièces, et chaque support contenait 5 racks.
Nous pouvons donc supposer que métalliser un support complet coûte \SI{280}{\chf}.

Au final, nous arrivons à une formule empirique pour l'étape de métallisation :

\begin{equation*}
    \text{Coût par pièce} = \frac{\SI{280}{\chf}}{\text{Nombre de pièces par bain}}
\end{equation*}


\subsection{Récapitulatif}
Le calcul des coût appliqué au couvercle d'exemple (fig. \ref{fig:example-part}) donne le résultat visible dans le tab. \ref{tab:cost-final}.
\begin{table}[h!]
\centering 
\begin{tabular}{l S[table-format=3.2] r} 
\toprule 
Injection & 1.34 & \si{\chf\per\piece} \\
Activation sélective & 0.47 & \si{\chf\per\piece} \\  
Métallisation & 0.70 & \si{\chf\per\piece} \\ 
\cmidrule(l){2-3}
\textbf{\textsc{Total}} & 2.51 & \si{\chf\per\piece} \\

\bottomrule 
\end{tabular}
\caption{Récapitulatif des coûts.} 
\label{tab:cost-final}
\end{table}




\begin{figure}[h]

    \begin{tikzpicture}
        \pie[sum=auto,text=legend, after number=\si{\chf}]{0.38/Injection, 0.48/Activation, 0.68/Métallisation}
    \end{tikzpicture}
    \begin{center}
        \caption{Pièce d'exemple. Les dimensions et la géométrie sont approximatives.}\label{fig:example-part}
    \end{center}
\end{figure}
