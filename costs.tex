\section{Étude de coûts}

\subsection{Activation Laser}
Pour calculer le coût de l'étape d'activation sélective (\textsc{lds}) nous avons utilisés les hypothèses de départ suivantes :
\begin{itemize}
    \item Le secteur d'activation laser tourne \SI{24}{\hour} par jour, 5 jours sur 7.
    \item La machine est une Microline 160i de chez LPKF coûtant \SI{220000}{\chf}.
        Elle est amortie en 5 ans.
    \item Un opérateur ne peut s'occuper que d'une machine à la fois.
    \item Le temps de cycle est de \SI{25}{\second\per\piece}.
        Ce temps change très peu suivant la complexité des connexions électriques car la majorité du temps est prise utilisée pour la mise en place de la pièce et l'alignement visuel par la machine.
    \item La machine consomme en permanence sa puissance de pointe soit \SI{2.5}{\kilo\watt} \cite{lpkf-microline-series}.
        L'électricité représentant une fraction faible du coût final, cette hypothèse n'induis pas de grande erreurs.
\end{itemize}

Ces hypothèses sont basées sur ce que nous avons eu l'occasion de voir lors de notre visite chez Cicorel.
Elle s'appliquent à la production d'un couvercle anti-ingénierie inverse pour lecteur de carte.
Le seul facteur qui change pour une autre pièce est le temps d'illumination laser, qui représente une fraction du temps de cycle.
Les résultats du tab. \ref{tab:cost-laser-activation} sont donc valables pour la pluspart des pièces.


\begin{table}[h!]
\centering 
\begin{tabular}{l S[table-format=3.2] r} 
\toprule 
Consommation électrique & 2.5 & \si{\kilo\watt} \\
Prix de l'électricité & 0.25 & \si{\chf\per\kilo\watt\per\hour} \\
\cmidrule(l){2-3}
Électricité & 0.625 & \si{\chf\per\hour} \\
\midrule
Prix de la machine & 220000 & \si{\chf} \\
Fonctionnement & 6240 & \si{\hour\per\annee} \\
\cmidrule(l){2-3}
Amortissement sur 5 ans & 7.05 & \si{\chf\per\hour} \\
\midrule
Opérateur & 60 & \si{\chf\per\hour} \\
\midrule
\midrule
Coût horaire & 67.68 & \si{\chf\per\hour} \\
Temps de cycle & 25 & \si{\second\per\piece} \\
\cmidrule(l){2-3}
\textbf{Total} & 0.47 & \si{\chf\per\piece} \\

\bottomrule 
\end{tabular}
\caption{Calcul des coûts de l'activation sélective par laser} 
\label{tab:cost-laser-activation}
\end{table}


